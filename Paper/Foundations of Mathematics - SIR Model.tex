\documentclass{amsart}
\usepackage[scaled]{helvet}
\renewcommand*\familydefault{\sfdefault} % These two commands set the base font of the document to sans serif

\usepackage{mathrsfs} %I don't know what this package is, but I was scared to delete is
\usepackage{amssymb} %Standard packages
\usepackage{amsmath}
\usepackage{amsthm}
\usepackage{sansmath} %This makes the math symbols match sans serif
\usepackage{color}
\usepackage{hyperref} %Makes hyperlinked Theorem, Figure, and Citation links
\usepackage{epsfig}
\usepackage{graphicx}
\usepackage{lipsum}  % Lorem Ipsum placeholder text for the outline.
\input xy
\usepackage{subfig}
\xyoption{all}

\usepackage{fancyhdr} %Package that allows for neat headers
\pagestyle{fancy}
\fancyhead[LE]{{\sc Proc. Stockton Math. Sem.}} %Header on the left of even numbered pages
\fancyhead[LO]{\textit{\leftmark}} %Header on the left of odd numbered pages
\fancyhead[RE]{\textit{\leftmark}} %Header on the right of even numbered pages
\fancyhead[RO]{\shorttitle} %Header on the right of odd numbered pages
\cfoot{\thepage} %Footer commands
% End of pile of commands that makes the header!

% FIRST PAGE NUMBER
\setcounter{page}{43}                  %%%%%%%
\setlength{\textwidth}{4.4in}          %%%%%%%
\setlength{\textheight}{7.0in}         %%%%%%%
\setlength{\evensidemargin}{1in}       %%%%%%%
\setlength{\oddsidemargin}{1in}        %%%%%%%
\setlength{\topmargin}{.8in}

\newtheorem{theorem}{Theorem}[section]
\newtheorem{lemma}[theorem]{Lemma}
\newtheorem{proposition}[theorem]{Proposition}
\theoremstyle{definition}
\newtheorem{definition}[theorem]{Definition}
%\newtheorem*{maintheorem}{Theorem \ref{thm:main}}
\newtheorem{remark}[theorem]{Remark}
\numberwithin{equation}{section}

%%%% HERE YOU CAN DEFINE YOUR FAVORITE SHORTCUT COMMANDS:
%
    % VECTORES (boldface)
    \def\ve#1{{\bf #1}}
    % NORMS OF THESE BEASTS:
    \def\norm#1{\| {\bf #1} \|}
    % JUST NORMS
    \def\norma#1{\| { #1} \|}
    % VECTORES CON FLECHITA: (december 2003)
    \def\vector#1{\overrightarrow{\strut{#1}}}
    % Rn
    \def\rn{{\Bbb R}^n}
    % Blackboard R
    \def\r{{\Bbb R}}
    % ABS value
    \def\abs#1{\vert \, {#1} \, \vert}

%%%%%%%%%%%%%%%%%%%%%%%%%%%%%%%%%%%%%%%%%%

\begin{document}

\begin{sansmath}

\begin{center}
\begin{picture}(256,56)
\put(63,37){\textbf{Proceedings of the Stockton}}
% \put(55,23){\textbf{University Mathematics Seminar}}
\put(79,23){\textbf{Mathematics Seminar}}
%\put(85,11){\tiny{Presented on [date of the presentation]}}
%\put(0,0){\framebox(255,54){}}
%\put(2,2){\framebox(251,50){}}
%\put(8,10){\includegraphics[scale = .55]{Stockton.png}}
%\put(210,10){\includegraphics[scale = .5]{Sigma.png}}
\end{picture}
\end{center}

\title[Running Title (shorter)]
{SIR Modeling and Covid}

\author{Steven Blythe}
\email{StevenBlytheSU@gmail.com}


\keywords{}

\begin{abstract}
ABSTRACT HERE

\end{abstract}

\maketitle

\section{Introduction}

What is Differential Equations and why do we use it to model infectious diseases? It is important to note that the goal of mathematical models are not to create a perfect representation of an event occurring in the real world. Instead, mathematical models are a mere representation of an event with the goal to help understand an aspect of the event. % 2

While we may use mathematical models, we must take note that the real world has many variables that complicates our representation. For example, we might make a model for the amount of crop yield in a given year based on crop data. However, we may say blueberries tend to bloom over the summer so we should expect a higher yield over the summer, real-world factors such as droughts, demand fluctuation, and ongoing diseases may lower crop yield in a given year.

When we build models, we should look to compare our model with data from the real world. If our model accurately describes what we are representing, then we have created a tool to assist us in predicting a system. Otherwise, if our model is wrong, we should study and pay close attention to our model to improve our knowledge for our assumption.

\section{Model Building}

Think to yourself. How do you think you might model population growth? If we have a population of $100$ in a given year, we might first consider adding a fraction of our population back into the total population. As the population grows, the fraction we add back in also grows. Our model may only have a few factors considered, but our model is a good basis for what we are looking for. Our model suggests the rate of growth of our population is proportional to the population itself. Using this assumption, we can derive a working model with variables.

We established a population for our model, a population of $100$, in our first model for population growth. Instead of using $100$, we can use variable $P$ to represent our population size. We also represented our population growth in terms of a fraction, or $k$. Here, we have a simple model to represent change from one instance. However, how should we find our population after many days? Here, we can use $t$ to represent years and $P(t)$ will represent the quantity of our population.

If we want to consider a model where we can find the rate of change of our population $P$, we might want to find $\frac{dP}{dt}$. Now, from our assumption, we are looking for a model where we can measure the rate of growth for a given population. We know this rate of growth is proportional to our population. Here, we want to look for the following differentual equation:

\[ \frac{dP}{dt} = kP \]

Here, we created a differential equation. Our differential equation is a first-order equation since we are only considering the first derivative of our dependent variable.

\section{Finding Meaning in our Models}

For the model we created, we know our model will predict our population after a given amount of time. Our model always assume our population will grow at a rate $k$. What would it mean if $\frac{dP}{dt} = 0$? Since our rate of growth is positive, that means our population $P$ must have began at $0$. Here, we can represent our equation as:

\[ \frac{dP}{dt} = k(0)\]

For all our $t$, $P(t) = 0$. Since $k$ is a positive constant, we can expect to have a population of $0$ if our population begins at $0$ at $t = 0$, or $t_0$. We can represent our initial population, or initial condition, as $P(t_0)$.

Can we expect our differential equation to be negative at some point? Since we are looking at a population, we only want to consider $kP \geq 0$, which would never 'grow' into a negative population.

Now, can we graph our equation right now? Not yet. If we graph $kP$, we only have a linear graph. Should our growth, which is proportional to our current population, produce a linear graph? If our growth is $10\%$ and our initial population is $100$, we may expect an increase of roughly $10$ people in one time step. If we move another time step, should we expect the same increase of $10$ or should we expect an increase of $11$? To find the equation we can use to predict our growth, we must solve our differential equation. If we solve our differential equation, our solution is as follows:

\[ P(t) = ce^{kt} \]

% Perhaps perform the solution?

This is the general solution for our system. Remember, $k$ is our rate of growth and $t$ represents time. Note that we also added $c$ to our equation. The constant $c$ came from an integration we performed to find our solution. To find $c$, we use our initial condition $P(t_0)$. If $P(t_0)$ is $10$, then we have:

\begin{align*}
  P(t_0) & = ce^{k(0)}\\
  10 & = c\\
  P(t) & = 10e^{kt}
\end{align*}

Here, we a specific solution for our initial condition, $P(t_0) = 10$. We can plot the specific solution to predict our population growth. Now, let us consider a few initial conditions for our starting population. Remember, population should be non-negative. Let us let $P(t_0)$ be $0$ first.

% Graph here | c = 0,

Looking at the graph, we can see our population is always $0$ for all of $t$. Now, let us consider a few other examples, such as $P(t_0) = 100, 200, 500$

% Graph here | c = 100, 200, 500

\section{Logistic Population Model}

Looking at the graphs in the previous section, should we expect our model to work in the real world? Not always. If we're modeling the population of rabbits in an area, we may expect an initial growth. However, as time goes on, there will eventually be not enough resources for all the rabbits in an area. Now, we want a model where we can predict a growth while under a threshold but a decay above a threshold. Let us look at our original equation for growth:

\[ \frac{dP}{dt} = kP \]

Here, let us consider a threshold for our equation, $N$. Let us consider a few cases surrounding $N$:

\begin{enumerate}
  \item If $P < N$, let population grow: $\frac{dP}{dt} \approx kP$
  \item If $P = N$, let our population growth become stagnant.
  \item If $P > N$, let population decrease: $\frac{dP}{dt} < 0$
\end{enumerate}

Here, we can see our population size and threshold are related in some way. If $P < N$, we should expect a positive number, whereas if $P > N$, we should expect a negative number. One way we can represent this is with the following:

\[ (1 - \frac{P}{N} )\]

Here, we can plug our new expression into our original equation.

\[ \frac{dP}{dt} = kP \Big(1 - \frac{P}{N} \Big) \]

% Produce the graph.

\section{Predator-Prey Sytems}
% Introduce a new variable.

% 1. Model rabbit population.
% 2. Introduce foxes into Rabbit's model
% 3. Model fox population
% 4. Introduce rabbits into fox's model

\section{Paul or SIR}


\newpage
% \begin{figure}[ht]
% \label{fig1}
% \centering
%{\includegraphics[height=30mm]{fig4.jpg}}
%\label{fig1}
% \qquad
%{\includegraphics[height=30mm]{fig5.jpg}}
% \caption{Smoothing hair on a flat head
% \label{fig1}}
% \end{figure}

% This optional version of the two figures side-by-side, in case you want
% to label the subfigures as (A) and (B),
% but still use a common label for both figures:
% (A) and (B) for every picture, framed
% (if you don't want the frames, remove both ``fbox''es)
% Keep in mind that sometimes pictures are not placed immediately
% after the text.
%
%    \begin{figure}[ht]
%    \centering
%    \subfloat[]{\fbox{\includegraphics[height=30mm]{fig4.jpg}}}
%    \qquad
%    \subfloat[]{\fbox{\includegraphics[height=30mm]{fig5.jpg}}}
%    \caption{Smoothing hair; (A): sticks out, (B): smooth.
%    \label{fig1optional}}
%    \end{figure}

% \medskip


    %_______________________________________________________
    %\centerline{ \includegraphics[height=35mm]{fig123.jpg}}
    %_______________________________________________________


The first figure can be referred to as  Figure \ref{fig1}.

Make sure to keep the figures in the same folder as the TeX document.
Acceptable formats are .jpg, .pdf, .png, and probably others as well.

Text continues here.

Sample citation:  \cite{alexandrov}.

More text.

\section{Our second section}

This is another section
with some math:
    $$ \ve u = (-x_2, x_1) $$


And more text.
    \medskip



\section{The main result}

This is another section.

With a couple more citations:
 \cite{alexandrov} and
 \cite{milnor}.


    \bigskip


\subsection{Milnor's proof}

This is a subsection.
It has a formula that we want numbered, and will refer to it later,
so we create a label for it:
    \begin{equation}
    \label{1}
    u(r \ve x) = r \ve u(\ve x) , \qquad \hbox{for $1 \le r \le 2$.}
    \end{equation}

And this is an example of how to refer to it: equation (\ref{1}).


This is something we want to call a Lemma:

\begin{lemma}
\label{lemma1}
    There exists a positive constant $C$ such that
    $$ \norma{\ve u(\ve x)  - \ve u  (\ve y))} \le C \norm{x-y} , $$
    for every $\ve x, \ve y$ in $A$.
\end{lemma}



\begin{proof}
Here goes the proof of the lemma.
\end{proof}

And this is how we refer to this result later:
As we proved in Lemma \ref{lemma1}, this is interesting.


\medskip

And this is something we want to call a Theorem:


\begin{theorem}
\label{hairy}
{\em(The Hairy Ball Theorem; smooth version.)}
    The sphere $S^2$ does not have a continuously differentiable vector field of unit tangent vectors.
\end{theorem}



\begin{proof}
  Here goes the proof of said theorem.
\end{proof}

We refer to this result as follows:
the nice result proved in Theorem \ref{hairy}.

\section{Say this is Section Four}

 Yet another section.
With some displayed formula:
    %
    \begin{equation}
    \label{stereo}
       \ve s (\ve x) = {1 \over \norm{x}^2 + 1} \, (2 x_1, 2x_2, \norm{x}^2 - 1) .
    \end{equation}
    %


  And we refer to it as: equation (\ref{stereo})

This is a footnote.
    \footnote{We clarify the above text here.}

    \medskip


\section{Appendix}

If you need an Appendix, it goes here.

\bigskip



And finally, the references you used (if any):

\bibliographystyle{plain}
\begin{thebibliography}{12}

\bibitem{milnor}
John Milnor, {\it
 American Mathematical Monthly}
    {\bf 85} (7), August-September 1978, 521--528.

 \smallskip

    \bibitem{alexandrov}
    P.~S.~Aleksandrov, {\it Combinatorial Topology},
    Dover Publications, 1998.


\end{thebibliography}

\newpage
\section{Paul's Online Notes}
https://tutorial.math.lamar.edu/Classes/DE/DE.aspx\\
\lipsum[0-2]

\section{Differential Equations by Paul Blanchard, Robert Devaney, Glen Hall}
\underline{Unlimited Growth}\\ % 4
\lipsum[2-4]\\
\underline{Logistic Population Models}\\ % 9
\lipsum[4-6]\\
\underline{Predator-Prey System}\\ % 12, 150
\lipsum[6-8]\\
\underline{The SIR Model of an epidemic}\\ % 210
\lipsum[8-10]

\section{The SIR Model for Spread of Disease}
https://www.maa.org/press/periodicals/loci/joma/the-sir-model-for-spread-of-disease-the-differential-equation-model\\
\lipsum[0-2]

\section{COVID-19 Futures, Explained with Simulations}
https://ncase.me/covid-19/\\
\lipsum[3-5]

\section{How Outbreaks Like Coronavirus Spread Exponentially, and How To "Flatten the Curve"}
https://www.washingtonpost.com/graphics/2020/world/corona-simulator/\\
\lipsum[1-3]

\section{Modeling Exponential Growth}
https://www.youtube.com/watch?v=Kas0tIxDvrg\\
\lipsum[2-3]

\section{Modeling COVID-19 with Differential Equations}
% https://julia.quantecon.org/continuous_time/seir_model.html \\
\lipsum[3-4]
\end{sansmath}

\end{document}
