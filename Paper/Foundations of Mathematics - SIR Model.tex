% Draft of template for seminar proceedings

\documentclass{amsart}

%\documentclass[12pt]{article}

%\setlength{\textwidth}{\paperwidth}
%\addtolength{\textwidth}{-1in}
%\calclayout

%\usepackage[margin=2cm]{geometry} %This command widens the margins

%\marginparwidth=70pt

\usepackage[scaled]{helvet}
\renewcommand*\familydefault{\sfdefault} % These two commands set the base font of the document to sans serif

\usepackage{mathrsfs} %I don't know what this package is, but I was scared to delete is
\usepackage{amssymb} %Standard packages
\usepackage{amsmath}
\usepackage{amsthm}
\usepackage{sansmath} %This makes the math symbols match sans serif
\usepackage{color}
\usepackage{hyperref} %Makes hyperlinked Theorem, Figure, and Citation links
\usepackage{epsfig}
\usepackage{graphicx}
\usepackage{lipsum}  % Lorem Ipsum placeholder text for the outline.
\input xy
\usepackage{subfig}
\xyoption{all}

%\usepackage[T1]{fontenc}
%\usepackage{euscript}


% This pile of commands makes the header!!!!!!
\usepackage{fancyhdr} %Package that allows for neat headers
\pagestyle{fancy}
\fancyhead[LE]{{\sc Proc. Stockton Math. Sem.}} %Header on the left of even numbered pages
\fancyhead[LO]{\textit{\leftmark}} %Header on the left of odd numbered pages
\fancyhead[RE]{\textit{\leftmark}} %Header on the right of even numbered pages
\fancyhead[RO]{\shorttitle} %Header on the right of odd numbered pages
\cfoot{\thepage} %Footer commands
% End of pile of commands that makes the header!

% FIRST PAGE NUMBER
\setcounter{page}{43}                   %%%%%%%
\setlength{\textwidth}{4.4in}          %%%%%%%
\setlength{\textheight}{7.0in}         %%%%%%%
\setlength{\evensidemargin}{1in}       %%%%%%%
\setlength{\oddsidemargin}{1in}        %%%%%%%
\setlength{\topmargin}{.8in}

\newtheorem{theorem}{Theorem}[section]
\newtheorem{lemma}[theorem]{Lemma}
\newtheorem{proposition}[theorem]{Proposition}
\theoremstyle{definition}
\newtheorem{definition}[theorem]{Definition}
%\newtheorem*{maintheorem}{Theorem \ref{thm:main}}
\newtheorem{remark}[theorem]{Remark}
\numberwithin{equation}{section}

%%%% HERE YOU CAN DEFINE YOUR FAVORITE SHORTCUT COMMANDS:
%
    % VECTORES (boldface)
    \def\ve#1{{\bf #1}}
    % NORMS OF THESE BEASTS:
    \def\norm#1{\| {\bf #1} \|}
    % JUST NORMS
    \def\norma#1{\| { #1} \|}
    % VECTORES CON FLECHITA: (december 2003)
    \def\vector#1{\overrightarrow{\strut{#1}}}
    % Rn
    \def\rn{{\Bbb R}^n}
    % Blackboard R
    \def\r{{\Bbb R}}
    % ABS value
    \def\abs#1{\vert \, {#1} \, \vert}

%%%%%%%%%%%%%%%%%%%%%%%%%%%%%%%%%%%%%%%%%%

\begin{document}

\begin{sansmath}

\begin{center}
\begin{picture}(256,56)
\put(63,37){\textbf{Proceedings of the Stockton}}
% \put(55,23){\textbf{University Mathematics Seminar}}
\put(79,23){\textbf{Mathematics Seminar}}
%\put(85,11){\tiny{Presented on [date of the presentation]}}
%\put(0,0){\framebox(255,54){}}
%\put(2,2){\framebox(251,50){}}
%\put(8,10){\includegraphics[scale = .55]{Stockton.png}}
%\put(210,10){\includegraphics[scale = .5]{Sigma.png}}
\end{picture}
\end{center}

\title[Running Title (shorter)]
{Title of Your Paper (long version)}

\author{Steven Blythe}
\email{StevenBlytheSU@gmail.com}


\keywords{}

\begin{abstract}
Here goes a short abstract of your paper.

\end{abstract}

\maketitle

\section{Introduction}
This is the introduction.
And this is an example of two figures side-by-side, with a common label:

\begin{figure}[ht]
% \label{fig1}
\centering
%{\includegraphics[height=30mm]{fig4.jpg}}
%\label{fig1}
\qquad
%{\includegraphics[height=30mm]{fig5.jpg}}
\caption{Smoothing hair on a flat head
\label{fig1}}
\end{figure}

% This optional version of the two figures side-by-side, in case you want
% to label the subfigures as (A) and (B),
% but still use a common label for both figures:
% (A) and (B) for every picture, framed
% (if you don't want the frames, remove both ``fbox''es)
% Keep in mind that sometimes pictures are not placed immediately
% after the text.
%
%    \begin{figure}[ht]
%    \centering
%    \subfloat[]{\fbox{\includegraphics[height=30mm]{fig4.jpg}}}
%    \qquad
%    \subfloat[]{\fbox{\includegraphics[height=30mm]{fig5.jpg}}}
%    \caption{Smoothing hair; (A): sticks out, (B): smooth.
%    \label{fig1optional}}
%    \end{figure}

Or only one Figure
(this simpler command can be used if we don't want a label,
or we don't want to refer to the figure later on):

\medskip


    %_______________________________________________________
    %\centerline{ \includegraphics[height=35mm]{fig123.jpg}}
    %_______________________________________________________


The first figure can be referred to as  Figure \ref{fig1}.

Make sure to keep the figures in the same folder as the TeX document.
Acceptable formats are .jpg, .pdf, .png, and probably others as well.

Text continues here.

Sample citation:  \cite{alexandrov}.

More text.

\section{Our second section}

This is another section
with some math:
    $$ \ve u = (-x_2, x_1) $$


And more text.
    \medskip



\section{The main result}

This is another section.

With a couple more citations:
 \cite{alexandrov} and
 \cite{milnor}.


    \bigskip


\subsection{Milnor's proof}

This is a subsection.
It has a formula that we want numbered, and will refer to it later,
so we create a label for it:
    \begin{equation}
    \label{1}
    u(r \ve x) = r \ve u(\ve x) , \qquad \hbox{for $1 \le r \le 2$.}
    \end{equation}

And this is an example of how to refer to it: equation (\ref{1}).


This is something we want to call a Lemma:

\begin{lemma}
\label{lemma1}
    There exists a positive constant $C$ such that
    $$ \norma{\ve u(\ve x)  - \ve u  (\ve y))} \le C \norm{x-y} , $$
    for every $\ve x, \ve y$ in $A$.
\end{lemma}



\begin{proof}
Here goes the proof of the lemma.
\end{proof}

And this is how we refer to this result later:
As we proved in Lemma \ref{lemma1}, this is interesting.


\medskip

And this is something we want to call a Theorem:


\begin{theorem}
\label{hairy}
{\em(The Hairy Ball Theorem; smooth version.)}
    The sphere $S^2$ does not have a continuously differentiable vector field of unit tangent vectors.
\end{theorem}



\begin{proof}
  Here goes the proof of said theorem.
\end{proof}

We refer to this result as follows:
the nice result proved in Theorem \ref{hairy}.

\section{Say this is Section Four}

 Yet another section.
With some displayed formula:
    %
    \begin{equation}
    \label{stereo}
       \ve s (\ve x) = {1 \over \norm{x}^2 + 1} \, (2 x_1, 2x_2, \norm{x}^2 - 1) .
    \end{equation}
    %


  And we refer to it as: equation (\ref{stereo})

This is a footnote.
    \footnote{We clarify the above text here.}

    \medskip


\section{Appendix}

If you need an Appendix, it goes here.

\bigskip



And finally, the references you used (if any):

\bibliographystyle{plain}
\begin{thebibliography}{12}

\bibitem{milnor}
John Milnor, {\it
 American Mathematical Monthly}
    {\bf 85} (7), August-September 1978, 521--528.

 \smallskip

    \bibitem{alexandrov}
    P.~S.~Aleksandrov, {\it Combinatorial Topology},
    Dover Publications, 1998.


\end{thebibliography}

\newpage
\section{Paul's Online Notes}
https://tutorial.math.lamar.edu/Classes/DE/DE.aspx\\
\lipsum[0-2]

\section{Differential Equations by Paul Blanchard, Robert Devaney, Glen Hall}
\underline{Unlimited Growth}\\ % 4
\lipsum[2-4]\\
\underline{Logistic Population Models}\\ % 9
\lipsum[4-6]\\
\underline{Predator-Prey System}\\ % 12, 150
\lipsum[6-8]\\
\underline{The SIR Model of an epidemic}\\ % 210
\lipsum[8-10]

\section{The SIR Model for Spread of Disease}
https://www.maa.org/press/periodicals/loci/joma/the-sir-model-for-spread-of-disease-the-differential-equation-model\\
\lipsum[0-2]

\section{COVID-19 Futures, Explained with Simulations}
https://ncase.me/covid-19/\\
\lipsum[3-5]

\section{How Outbreaks Like Coronavirus Spread Exponentially, and How To "Flatten the Curve"}
https://www.washingtonpost.com/graphics/2020/world/corona-simulator/\\
\lipsum[1-3]

\section{Modeling Exponential Growth}
https://www.youtube.com/watch?v=Kas0tIxDvrg\\
\lipsum[2-3]

\section{Modeling COVID-19 with Differential Equations}
https://julia.quantecon.org/continuous_time/seir_model.html \\
\lipsum[3-4]
\end{sansmath}

\end{document}
